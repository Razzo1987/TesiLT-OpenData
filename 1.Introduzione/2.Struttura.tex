\section{Struttura}
Il secondo capitolo parler� di cosa sono gli Open Data, cercando di darne una definizione che includa le varie sfaccettature di questa nuova pratica e filosofia, del loro ruolo nella societ� e della loro importanza in ambito governativo. Si affronter� l�analisi di chi sono gli attori principali intorno a questo argomento e le loro esigenze. Successivamente si affronter� l�aspetto legale riguardante la materia Open Data, dalla spinta iniziale Americana, fino alle leggi emanate negli ultimi anni nella nostra Penisola. Si far� inoltre un�analisi dell�attuale situazione Italiana per comprendere la differente situazione nelle varie regioni.\\

Nel terzo capitolo si approfondiranno alcune soluzioni tecnologiche, cercando di comprendere la loro architettura, analizzando le caratteristiche offerte e spiegando l�esperienza dell�utente finale.\\

All�interno del quarto capitolo si esaminer� la situazione del Comune di Vigevano, studiando un�adeguata soluzione rispetto all�esigenza di pubblicare le informazioni presenti nelle loro banche dati e alla volont� di integrare la piattaforma che verr� realizzata con quella in dotazione a Regione Lombardia.\\

Il quinto capitolo descriver� la realizzazione della piattaforma, partendo dall�installazione della piattaforma scelta e proseguendo con la personalizzazione e la creazione delle funzionalit� richieste in fase di progettazione.\\

Nel sesto capitolo sar� fatta una presentazione dell�utilizzo della piattaforma realizzata, cercando di mostrare tutte le sue potenzialit�.\\

Infine nel settimo capitolo sar� fatta un�analisi del lavoro svolto e saranno tratte le dovute conclusioni e ipotizzati gli sviluppi futuri del progetto.