\section{Descrizione del Comune}
% Storia
La citt� di Vigevano si trova a 30 chilometri da Milano, tra i boschi del Ticino, che insieme agli altri canali che l'attraversano offre magnifici scorci e oasi naturali.

La citt� ha la sua origine come luogo fortificato longobardo dove ora sorge il cortile del Castello, successivo � lo sviluppo del borgo esterno, oggi occupato dalla Piazza Ducale.

A partire dal 1198 si trasforma in un libero Comune, mentre nel 1277 la storia delle potenti famiglie milanesi dei Visconti prima e degli Sforza si lega a quella di Vigevano. Grazie all'opera di Luchino Visconti e di Ludovico Sforza detto il Moro, tra XIV e XV secolo, il borgo di Vigevano inizia la sua trasformazione in residenza estiva: il Castello viene adibito a dimora di prestigio grazie all'opera di artisti come Bramante e la Piazza Ducale liberata da case ed edifici diviene il regale atrio d'ingresso al Castello.

Nel 1530 Vigevano ottiene il titolo di citt� con una propria sede vescovile.\\

\textbf{Piazza Ducale}, una delle piazze pi� belle d'Italia, fu ideata e decorata dal Bramante con il concorso di Leonardo da Vinci, che non partecip� direttamente ai lavori, ma lasci� disegni e testimonianze scritte nei codici d'appunti. La piazza funge da ingresso d'onore al Castello. I lavori iniziarono nel 1492 e si conclusero nel 1494. Piazza Ducale rappresenta uno dei primi modelli di piazza rinascimentale sul modello di \"{}forum\"{} romano e uno dei migliori esempi dell'architettura lombarda del XV secolo. Si presenta come un rettangolo di 134 metri di lunghezza e 48 di larghezza edificato su tre lati. La forma architettonica � opera del vescovo-architetto Juan Caramuel Lobkowitz che chiuse il quarto lato con la facciata barocca della Chiesa Cattedrale.
Sotto i portici le botteghe, un tempo occupate dai commercianti di lana e seta, oggi offrono ai visitatori svariate occasioni di svago.\\

\textbf{Il castello} sorge nella parte pi� alta di Vigevano e costituisce una piccola citt� nella citt� essendo uno dei pi� grandi d'Europa. Anche alla sua realizzazione contribuirono Bramante e Leonardo che contribuirono a dargli il fascino rinascimentale che ancor oggi conserva. Tra il 1492 e il 1494 i lavori furono terminati e le lussuose sale interamente affrescate, in modo da poter ospitare la corte ducale e i loro ospiti, quali il re di Francia Carlo VIII e, pi� avanti l'imperatore Carlo V.
Con la fine della dinastia sforzesca (1535) il castello pass� agli spagnoli e inizi� un lento declino che lo vide ospitare solo eserciti e trasformarsi in caserma fino al 1960.
Da quella data � iniziato il restauro ed il Castello si sta trasformando in una cittadella dell'arte e della cultura grazie ai suoi musei visitabili - tra cui il curioso e unico in Italia Museo della calzatura - e a un ricco programma di mostre ed eventi musicali nel corso di tutto l�anno.
\cite{web-Comune-Vigevano}

\subsection{Servizio Informatico Comunale}
% Organizzativo sistemi informativi
