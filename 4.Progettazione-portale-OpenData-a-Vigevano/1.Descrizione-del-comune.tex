\section{Descrizione del Comune}
% Storia
La citt� di Vigevano si trova a 30 chilometri da Milano, tra i boschi del Ticino. 

%-------------
Circondata dai boschi del Parco del Ticino, a soli 30 chilometri da Milano, Vigevano accoglie il visitatore con l'armonia della celebre Piazza Ducale: \"{}una sinfonia su quattro lati\"{} secondo la definizione del grande maestro Arturo Toscanini.

Ideata dal Bramante con il concorso di Leonardo da Vinci, Piazza Ducale � l'ingresso d'onore all'imponente Castello, per estensione uno dei pi� grandi d'Europa e in fase avanzata di restauro e di riuso grazie all'organizzazione di mostre e alla prossima apertura di musei.

Citt� d'arte ma anche citt� d'acque, Vigevano � attraversata da canali e dal fiume Ticino che offre scorci e oasi naturali di indubbio fascino.

Da Ludovico il Moro a Eleonora Duse, allo scrittore Lucio Mastronardi, sono tanti i personaggi che hanno visto la luce in una citt�, ancora oggi nota in tutto il mondo per la produzione di scarpe di qualit� e per la sua industria meccano-calzaturiera.

\subsection{Origini}
Di origine longobarda Vigevano nasce come luogo fortificato corrispondente all'attuale cortile del Castello. Successivamente si sviluppa il borgo esterno, le cui case ed edifici sorgono sul luogo oggi occupato dalla famosa Piazza Ducale.

Si trasforma in libero Comune a partire dal 1198, mentre nel 1277 la storia di Vigevano si lega a quella delle potenti famiglie milanesi dei Visconti prima e degli Sforza poi.

Grazie all'opera di Luchino Visconti e di Ludovico Sforza detto il Moro, tra XIV e XV secolo, il borgo di Vigevano inizia la sua trasformazione in residenza estiva, in delizioso soggiorno per gli svaghi e gli ozi della corte ducale: il Castello viene adibito a dimora di prestigio grazie all'opera di artisti come Bramante, la Piazza Ducale in scenografico spazio libero da case ed edifici, regale atrio d'ingresso al Castello

Nel 1530 Vigevano ottiene il titolo di citt� con una propria sede vescovile.



%- Organizzativo sistemi informativi
%Dovrebbe essere sul loro sito
