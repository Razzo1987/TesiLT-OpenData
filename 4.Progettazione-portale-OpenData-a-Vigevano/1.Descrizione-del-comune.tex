\section{Descrizione del Comune}
% Storia
Vigevano � un comune italiano di circa 60.000 abitanti della provincia di Pavia in Lombardia, dista circa 38 km da Pavia, 36 km dal centro di Milano. Il comune � il secondo della provincia per numero di abitanti dopo il capoluogo e primo per superficie. La citt� si trova nei pressi del Parco naturale lombardo della Valle del Ticino e la fitta vegetazione e i canali che attraversano la citt� offre al visitatore magnifici scorci e oasi naturali.

La citt� ha la sua origine come luogo fortificato longobardo dove ora sorge il cortile del castello, successivo � lo sviluppo del borgo esterno, oggi occupato dalla Piazza Ducale.

A partire dal 1198 si trasforma in un libero Comune, mentre nel 1277 la storia delle potenti famiglie milanesi dei Visconti e degli Sforza si lega a quella di Vigevano. Grazie all'opera di Luchino Visconti e di Ludovico Sforza, detto il Moro, tra il XIV e il XV secolo, il borgo di Vigevano inizia la sua trasformazione in residenza estiva: il castello viene adibito a dimora di prestigio, grazie all'opera di artisti come Bramante, e la Piazza Ducale, liberata da case ed edifici, diviene il regale atrio d'ingresso al castello.

Nel 1530 Vigevano ottiene il titolo di citt� con una propria sede vescovile.\\

%\textbf{Piazza Ducale}, una delle piazze pi� belle d'Italia, fu ideata e decorata dal Bramante con il concorso di Leonardo da Vinci, che non partecip� direttamente ai lavori, ma lasci� disegni e testimonianze scritte nei codici d'appunti. La piazza funge da ingresso d'onore al Castello. I lavori iniziarono nel 1492 e si conclusero nel 1494. Piazza Ducale rappresenta una delle prime piazze rinascimentali sul modello di �forum� romano e uno dei migliori esempi dell'architettura lombarda del XV secolo. Si presenta come un rettangolo di 134 metri di lunghezza e 48 di larghezza, edificato su tre lati. La forma architettonica � opera del vescovo-architetto Juan Caramuel Lobkowitz, che chiuse il quarto lato con la facciata barocca della Chiesa Cattedrale.
%Sotto i portici, le botteghe, un tempo occupate dai commercianti di lana e seta, oggi offrono ai visitatori svariate occasioni di svago.\\

%\textbf{Il castello} sorge nella parte pi� alta di Vigevano e costituisce una piccola citt� nella citt�, essendo uno dei pi� grandi d'Europa. Anche alla sua realizzazione parteciparono Bramante e Leonardo che contribuirono a dargli il fascino rinascimentale che ancor oggi conserva. Tra il 1492 e il 1494 i lavori furono terminati e le lussuose sale interamente affrescate, in modo da poter ospitare la corte ducale e i loro ospiti, quali il re di Francia Carlo VIII e, pi� avanti, l'imperatore Carlo V.
%Con la fine della dinastia sforzesca (1535) il castello pass� agli spagnoli e inizi� un lento declino che lo vide ospitare solo eserciti e trasformarsi in caserma fino al 1960.
%Da quella data � iniziato il restauro ed il Castello si sta trasformando in una cittadella dell'arte e della cultura grazie ai suoi musei visitabili - tra cui il curioso e unico in Italia Museo della calzatura - e a un ricco programma di mostre ed eventi musicali nel corso di tutto l�anno.
%\cite{ComuneVigevano-CenniStorici}

\subsection{Servizio Informatico Comunale}
% Organizzativo sistemi informativi
Durante la progettazione del portale OpenData per il Comune di Vigevano, il Dipartimento di Informatica Sistemistica e Comunicazione (DISCo) dell'Universit� di Milano Bicoca si � relazionato con il responsabile del Servizio Informatico Comunale (SIC) \href{malto:obaracchi@comune.vigevano.pv.it}{\textbf{Oscar Baracchi}}.\\

Il SIC � la struttura organizzativa cui compete la gestione del sistema informativo dell�Ente (il Comune di Vigevano), attraverso alcune decine di server con supporto ad oltre 350 utenti utilizzatori di apparecchiature informatiche, appositamente configurate per l'accesso alle informazioni presenti nelle banche dati dell�Ente ed alle procedure per l�erogazione di servizi ai cittadini ed alle imprese.

Il Servizio cura la pianificazione, lo sviluppo, il mantenimento, il coordinamento ed il controllo di tutte le iniziative ed attivit� che afferiscono i sistemi informativi comunali, le infrastrutture informatiche, la rete trasmissione dati, la conduzione di progetti nel campo dell�ICT di notevole complessit� tecnologica ed organizzativa.

In particolare il Servizio ha la responsabilit�:
\begin{itemize}
\item della pianificazione strategica e tattica per tutti gli aspetti relativi all�utilizzo dell�ICT (Information \& Communication Technology) nell�Amministrazione Comunale;
\item dello sviluppo di nuove iniziative ed attivit� per il miglioramento del grado di efficienza ed efficacia dell�azione amministrativa tramite l�utilizzo di opportuni sistemi informativi, infrastrutture informatiche e telematiche;
\item del mantenimento in efficienza dei sistemi informativi comunali, delle infrastrutture ed apparecchiature informatiche, di rete trasmissione dati utilizzate nell�Amministrazione comunale;
\item di centro acquisitore per quanto riguarda le forniture di beni e servizi che rientrino nell�ambito dell�ICT;
\item del supporto nella gestione di progetti ed attivit� in cui sia presente una componente informatica o telematica;
\item della consulenza e supporto alle unit� organizzative dell�Amministrazione comunale su aspetti che attengono in qualche misura all�ICT;
\item del supporto nella gestione di progetti ed attivit� di e-government e sovracomunali;
\item del supporto nei rapporti con i fornitori di servizi, tecnologie e soluzioni;
\end{itemize}
Inoltre sono stati sviluppati e manutenuti in proprio, gi� da anni, con tecnologia web (Java-Jsp) su database MySql ed application server Tomcat/Apache, numerosi applicativi.
\cite{ComuneVigevano-SIC}


Il SIC lavora da tempo ha adottando soluzioni Open Source per le piattaforme e i servizi offerti dal Comune, inoltre ha manifestato un forte interessa alla pubblicazione di diverse informazioni in formato Open Data e alla realizzazione di una piattaforma che ne permetta la divulgazione.