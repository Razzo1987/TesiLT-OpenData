\chapter*{Ringraziamenti}
\addcontentsline{toc}{chapter}{Ringraziamenti}
\chaptermark{Ringraziamenti}

Non � facile citare e ringraziare, in poche righe, tutte le persone che hanno contribuito alla nascita e allo sviluppo di questa tesi di laurea. � stato un lungo e piacevole cammino, che ha contribuito alla mia crescita personale.\\

Tutto � partito con un treno diretto a Bologna per andare ad un raduno che gli organizzatori stessi definivano essere un \textit{�incontro in carne, ossa e tortellini�}. In questa giornata ho avuto il tempo di parlare approfonditamente con il mio compagno di viaggio e l�occasione di conoscerlo meglio, nascendo cos� in me un sincero sentimento di fiducia ed amicizia. Nella stessa occasione ho scoperto l�esistenza di \href{http://www.spaghettiopendata.org/}{\texttt{Spaghetti Opendata}}, un gruppo di cittadini interessati al rilascio di dati pubblici in formato aperto, ma anche una comunit� viva ed accogliente. A loro va il mio primo ringraziamento, perch� da quel 18 Gennaio 2013 ho aperto gli occhi su una nuova realt� e gli Open Data hanno iniziato a ronzare nella mia testa.\\

Un sentito ringraziamento va anche ai miei genitori, che hanno deciso di supportarmi (o forse sopportarmi) in questi ultimi 4 anni, dopo la decisione di punto in bianco di cambiare il Corso di Laurea, abbandonando quello a cui ero iscritto da 3 anni. Non � stato semplice, � stata una nuova avventura, ma che ha dato finalmente i suoi frutti! Sono riusciti a starmi vicino anche in questa difficile scelta, nonostante inizialmente non condividessero a pieno.\\
 
In questi anni ho anche avuto l�occasione di entrare in un altra grande famiglia, sono entrato timido timido con le orecchie tese ad ascoltare, fino ad arrivare oggi a sentirmi parte attiva di essa. Penso che il Consiglio di Coordinamento Didattico di Scienze e Tecnologie Informatiche sia una raro esempio di come le idee dei singoli possano essere portate con serenit� su un tavolo di discussione e di confronto. Per questo motivo voglio ringraziate tutti i componenti del CCD e l�intero Dipartimento di Informatica per avermi trattato come un figlio e avermi dato l�opportunit� di crescere.\\

Molti ringraziamenti vanno a coloro che hanno condiviso con me anche solo un piccolo pezzo del mio cammino in quest�Universit�:
\begin{quoting}
Abbiamo incontrato la gente pi� strana\\
e imbarcato compagni di viaggio\\
qualcuno � rimasto\\
qualcuno � andato e non s'� pi� sentito.\\
 \\
Buon viaggio hermano querido \footnote{�caro fratello� in Spagnolo}\\
e buon cammino ovunque tu vada\\
forse un giorno potremo incontrarci\\
di nuovo lungo la strada.\\
 \\
Che le stelle ti guidino sempre\\
e la strada ti porti lontano! \footnote{ strofe (alcune riadattate) di �La Strada� dei Modena City Ramblers}
\end{quoting}

Come dimenticare i compagni di bevute serali e i produttori di buoni dosi di spritz, che sono riusciti ad alleviare il mio stress nei momenti in cui nulla sembrava funzionare. Senza di voi ne sarei uscito pi� pazzo di quanto gi� lo sia!\\

Infine un enorme grazie a tutti coloro che hanno avuto l�ardire di leggere questa tesi, amici e compagni, grazie di cuore per i suggerimenti e il supporto che mi avete dato nella scrittura di questa tesi, di cui mi sento veramente orgoglioso, grazie a chi si � interessato a una tematica che � diventata a me molto cara, a chi ha anche solo provato a capirci qualcosa. Spero che almeno un granello dell�entusiasmo che mi accompagna nel sentire la parola Open Data abbia potuto travolgervi.