\section{Impianto Normativo}

Il percorso che ha portato all�attuale normativa in ambito Open Data � lungo e intreccia le sue radici con i concetti di �libert� di informazione� e �trasparenza della pubblica amministrazione� ed ha interessato i governi di differenti nazioni.

\begin{itemize}
\item 4 luglio 1966, viene emanato negli USA il \textbf{Freedom of Information Act} (FOIA) \cite{foia}, �atto per la libert� di informazione�, � una legge sulla libert� di informazione, che impone alle pubbliche amministrazioni una serie di regole per permettere a chiunque di sapere come opera il Governo federale.
Il FOIA ha aperto a giornalisti e studiosi l'accesso agli archivi di Stato statunitensi, a molti documenti riservati e coperti da segreto di Stato, di carattere storico o di attualit�. Il provvedimento � stato un punto importante per garantire la trasparenza della pubblica amministrazione nei confronti del cittadino, il diritto di cronaca e la libert� di stampa dei giornalisti.

%Rapporto Mandelkern:  D. Mandelkern,Diffusion des donn�es publiques et r�volution num�rique Commissariat g�n�ral du Plan, 1999 -> Dati pubblici essenziali

%Libro verde sull�informazione del settore pubblico nella societ� dell�informazione� della Commissione europea, 2000 

\item 24 novembre 2000, viene pubblicata nella Gazzetta Ufficiale le \textbf{Disposizioni per la delegificazione di norme e per la semplificazione di procedimenti amministrativi} (legge 340/2000) \cite{legge340/2000}
\begin{quoting}
Le pubbliche amministrazioni di cui all'articolo 1, comma 2, del decreto legislativo n. 29 del 1993 hanno accesso gratuito ai dati contenuti in pubblici registri, elenchi, atti o documenti da chiunque conoscibili.
\cite{legge340/2000}
\end{quoting}

%Data Quality Act � USA � 2001 (quality, objectivity, utility, and integrity of information)
	%https://en.wikipedia.org/wiki/Data_Quality_Act
	%http://rationalwiki.org/wiki/Data_Quality_Act
	%http://www.gpo.gov/fdsys/pkg/PLAW-106publ554/html/PLAW-106publ554.htm

\item 17 novembre 2003, viene approvata  del Parlamento europeo e del Consiglio la \textbf{Direttiva relativa al riutilizzo dell�informazione del settore pubblico} (Direttiva PSI) \cite{direttiva-PSI} che incoraggia gli enti pubblici degli stati membri a massimizzare il potenziale dell'informazione rendendo disponibili e favorendo il riuso dei documenti, attraverso indici on line e licenze standard.

%Iniziative della Autorita� per la Informatica nella PA, anni 2000-2003.

\item 24 gennaio 2006, l�attuazione italiana della direttiva europea avviene con il \textbf{Decreto legislativo 24 gennaio 2006} \cite{decreto-legislativo2006}. Il provvedimento � stato predisposto dal Ministro per le politiche comunitarie e da quello per l'innovazione e le tecnologie, in accordo con i dicasteri degli Affari Esteri, Giustizia, Economia e Finanze, Funzione pubblica.
Esso definisce che il titolare del dato predispone le licenze standard per il riutilizzo e le rende disponibili, ove possibile in forma elettronica, sui propri siti istituzionali.
Gli Enti pubblici possono richiedere un compenso in denaro: in questo caso hanno l'obbligo di fissare e pubblicare le tariffe e le relative modalit� di versamento da corrispondere a fronte delle attivit�, individuandole sulla base dei costi effettivi sostenuti dalle Amministrazioni e aggiornato ogni due anni, comprende i costi di raccolta, di produzione, di riproduzione e diffusione maggiorati, nel caso di riutilizzo per fini commerciali.
� fatto divieto di esclusivit� sull�utilizzo dei dati.

\end{itemize}






%slides maurino parla di Nota 12 giugno 2008

Legge brunetta 150/2009
%slides maurino


Articolo 18 decreto sviluppo 2012
%prima dell'art.18 del DL 83/2012 si parla di OpenFatture, poi chiamato "Dentro il bilancio"
%slides maurino

Articolo 52 cad 2012
%slides maurino (nuovo CAD)	
\item 
Il Codice dell'Amministrazione Digitale, all'art. 52, comma-1-bis, prevede che le Pubbliche Amministrazioni debbano promuovere "progetti di elaborazione e di diffusione dei dati pubblici di cui sono titolari", nonch� assicurarne la pubblicazione "in formati aperti", al fine di "valorizzare e rendere fruibili" i dati stessi.

Tale impostazione ha trovato autorevole conferma nel "Vademecum sull'OpenData" pubblicato dal Ministero per la Pubblica Amministrazione e l'Innovazione nel mese di ottobre 2011.
http://www.dati.gov.it/sites/default/files/VademecumOpenData.pdf


Open data by default 2012
% forse 2013
%Slides maurino Open data by default

%http://www.morenaragone.it/?p=316
	%decreto-legge 22 giugno 2012, n. 83
	%decreto-legge 12 ottobre 2012, n. 179
	%legge 6 novembre 2012, n. 190

%Con l'entrata in vigore del D.Lgs. n. 33 /2013 "Riordino della disciplina riguardante gli obblighi di pubblicit�, trasparenza e diffusione di informazioni da parte delle pubbliche amministrazioni", l�Amministrazione comunale sta provvedendo all�adeguamento delle pagine del sito web.


% 5o7 giugno direttiva PSI


%leggi regionali...
	%http://www.dati.piemonte.it/novita/1-ultime/875-la-nuova-direttiva-europea-in-materia-di-psi.html