\section{Movimento OpenData}
% Filosofia, scala 5 stelle, Obama, ...

Con Open Data si definisce\cite{uno} aaaa.
%Con Open Data si definisce una filosofia (ma anche una pratica) che implica che alcune tipologie di dati siano liberamente accessibili a tutti, senza restrizioni di copyright, brevetti o altre forme di controllo che ne limitino la riproduzione. L'open data si richiama alla più ampia disciplina dell’Open Government, cioè una dottrina che prevede l’apertura della Pubblica amministrazione, intesa sia in termini di trasparenza che di partecipazione diretta dei cittadini, anche attraverso l’uso delle nuove tecnologie della comunicazione”. Partecipazione, Democrazia, Comunità, Diritti… sono solo alcune delle parole che vengono associate all’OpenData. Pur abbracciando l’aspetto filosofico e concordando con il comune approccio ideologico (i dati sono acquisiti con i soldi dei cittadini e a loro debbono tornare !), riteniamo che solo un’analisi critica ed impietosa della situazione attuale possa produrre risultati concreti (si veda il percorso travagliato dell’OpenSource all’interno della Pubblica Amministrazione)


%I dati aperti, comunemente chiamati con il termine inglese open data anche nel contesto italiano, sono alcune tipologie di dati liberamente accessibili a tutti, privi di brevetti o altre forme di controllo che ne limitino la riproduzione e le cui restrizioni di copyright eventualmente si limitano ad obbligare di citare la fonte o al rilascio delle modifiche allo stesso modo. L'open data si richiama alla più ampia disciplina dell’open government, cioè una dottrina in base alla quale la pubblica amministrazione dovrebbe essere aperta ai cittadini, tanto in termini di trasparenza quanto di partecipazione diretta al processo decisionale, anche attraverso il ricorso alle nuove tecnologie dell'informazione e della comunicazione; e ha alla base un'etica simile ad altri movimenti e comunità di sviluppo "open", come l'open source, l'open access e l'open content. Nonostante la pratica e l'ideologia che caratterizzano i dati aperti siano da anni ben consolidate, con la locuzione "open data" si identifica una nuova accezione piuttosto recente e maggiormente legata a Internet come canale principale di diffusione dei dati stessi.
